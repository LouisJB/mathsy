\documentclass[11pt]{article}   	% use "amsart" instead of "article" for AMSLaTeX format
\usepackage{geometry}                		% See geometry.pdf to learn the layout options. There are lots.
\geometry{a4paper, margin=1in}                % ... or a4paper or a5paper or ... 
%\geometry{landscape}                		% Activate for for rotated page geometry

%\usepackage{enumitem}
%\parskip = \baselineskip
%\setlength{\parskip}{5pt}
%\setlength{\parindent}{0pt}
\usepackage[parfill]{parskip}    			% Activate to begin paragraphs with an empty line rather than an indent

%\usepackage{enumitem}

\usepackage{graphicx}				% Use pdf, png, jpg, or eps§ with pdflatex; use eps in DVI mode

\usepackage{amsfonts}								% TeX will automatically convert eps --> pdf in pdflatex		
\usepackage{amssymb}
\usepackage{amsmath}

\usepackage{hyperref}

\setlength\parindent{0pt}			% globally suppress indentation

\usepackage{empheq}
\usepackage[most]{tcolorbox}

\usepackage{changepage}   % for the adjustwidth environment

\newtcbox{\mymath}[1][]{%
    nobeforeafter, math upper, tcbox raise base,
    enhanced, colframe=blue!30!black,
    colback=blue!30, boxrule=1pt,
    #1}

\newenvironment{monoblock}%
  {\ttfamily \small}%
  {}
  
\newenvironment{tight_enumerate}{
\begin{enumerate}
  \setlength{\itemsep}{0pt}
  \setlength{\parskip}{0pt}
}{\end{enumerate}}


\title{M347 TMA02 R2698663}
\author{Louis Botterill}
\date{07-01-2018}					% Activate to display a given date or no date


\begin{document}
\maketitle

\pagebreak

\section*{Question 1}

\subsection*{1.a}

$ x_1, x_2, ..., \ x_n $ are independent observations taken from a Pareto distribution with pdf \\
\\
\[ f(x|\beta) = \frac{\beta}{x^{\beta + 1}} \text{ on } x > 1 \text{; where } \beta > 0 \]
\\
We can show the log-likelihood is as required as follows \\
\\
\textbf{Definition:} Log-likelihood is defined as: \\
\\
\begin{empheq}[box={\mymath[colback=blue!8,drop lifted shadow, sharp corners]}]{equation*}
  l(\theta) = log \ L(\theta) = \sum_{i=1}^{n} \ log \ f(x_i | \theta)
\end{empheq}
So we have \\
\begin{flalign*}
l(\beta) &= \sum_{i=1}^{n} \ log \ \frac{\beta}{x_i^{\beta + 1}} \\
l(\beta) &= n \ log \ \beta + \sum_{i=1}^{n} \ log \ \frac{1}{x_i^{\beta + 1} } \\
l(\beta) &= n \ log \ \beta - \sum_{i=1}^{n} \ log \ x_i^{\beta + 1} \\
l(\beta) &= n \ log \ \beta - (\beta + 1) \sum_{i=1}^{n} \ log \ x_i \\
\end{flalign*}


\subsection*{1.b}

Find $ l'(\beta) $ and hence the candidate MLE
\\
\begin{flalign*}
  l'(\beta) &= \frac{d}{d \beta} l(\beta) = d \ [ log \ \beta - (\beta + 1) \sum_{i=1}^{n} \ log \ x_i ] / d \beta \\
  l'(\beta) &= \frac{n}{\beta} - \sum_{i=1}^{n} \ log \ x_i \\
\end{flalign*}
Solving for the stationary point where $ l'(\theta) = 0 $ \\
\\
\[ \frac{n}{\beta} - \sum_{i=1}^{n} \ log \ x_i = 0 \] \\
\[ \frac{n}{\beta} = \sum_{i=1}^{n} \ log \ x_i \] \\
\\
So the candidate MLE is
\\
\[ \hat{ \beta } = \frac{n}{\sum_{i=1}^{n} \ log \ x_i } \] \\
\\
(At this point $ \hat{\beta} $ it is only a candidate MLE of $ \beta $ since we've not proved $ \hat{\beta} $ is a maximum, it could be a saddle or minimum.) \\
\\


\subsection*{1.c}

We can confirm that $ \hat{\beta} $ is indeed the MLE of $ \beta $ by showing that the stationary point of the function $ l'(\beta) $ is a maximum. This occurs when the second derivative $ l''(\beta) < 0 $
\\
\begin{flalign*}
  l'(\beta) &= n \beta^{-1} - \sum_{i=1}^{n} \ log \ x_i  \\
  l''(\beta) &= - n \beta^{-2} = - \frac{n}{\beta^{2} }  \\
\end{flalign*}
Since $ n > 0 $ (and $ \beta > 0 $), then $ l''(\beta) < 0 $, for all $ n, \beta $ \\
\\
Hence, $$ \hat{ \beta } = \frac{n}{\sum_{i=1}^{n} \ log \ x_i } $$ is an MLE of $ \beta $ \\
\\


\subsection*{1.d}

Explain why $ \hat{\beta} > 0 $, as one would hope given that $ \beta > 0 $ \\
\\
We have found that the MLE $ \hat{\beta} $ is defined as \\
\\
\[ \hat{ \beta } = \frac{n}{\sum_{i=1}^{n} \ log \ x_i } \] \\
\\
where $ n > 0 $, positive ($ n = 0 $ would be a degenerate case) \\
\\
and also $ x > 1 $, positive (from the definition of the distribution) \\ 
\\
\textbf{Definition:} By definition of logarithms
\\
\begin{empheq}[box={\mymath[colback=blue!8,drop lifted shadow, sharp corners]}]{equation*}
  ln(x > 1) > 0 
\end{empheq} \\
\\
and so the result of positive n times 1/sum of positive values is always positive, as required \\

\break 


\section*{Question 2}

$ x_1, x_2 \ ... \ x_n $ are independent observations taken from a Pareto distribution with pdf \\
\\
\[ f(x|\beta) = \frac{\beta}{x^{\beta + 1}} \text{ ; on } x > 1 \text{ where } \beta > 0 \]
\\
\[ l(\beta) = n \ log \ \beta - (\beta + 1) \sum_{i=1}^{n} \ log \ x_i \]
\\
\[ \hat{ \beta } = \frac{n}{\sum_{i=1}^{n} \ log \ x_i } \]
\\
Given
\\
\[ E \big\{-l''(\beta) \big\} = \frac{n}{\beta^2} \]
\\
and \\
\\
The hypothesis test is: $ H_0 : \beta = 1 $ against $ H_1 : \beta \neq 1. $ \\
\\
So $ \beta_0 = 1 $ and $ \beta_1 \neq 1 $ \\


\subsection*{2.a}

Obtain formulae for both Wald statistics, $ W_1 $ and $ W_2 $, in terms of $ \beta $ (and n).\\
\\
\textbf{Definition:} The Wald $ W_1 $ and $ W_2 $ statistics on parameter $ \theta $ \\
\begin{empheq}[box={\mymath[colback=blue!8,drop lifted shadow, sharp corners]}]{equation*}
\begin{split}
W_1 &= (\theta_0 - \hat{\theta})^2 E \big\{-l''(\theta) \big\}_{\hat{\theta}} \\
W_2 &= (\theta_0 - \hat{\theta})^2 E \big\{-l''(\theta) \big\}_{\theta_0}
\end{split}
\end{empheq}
So we have \\
\begin{flalign*}
W_1 &= (1 - \hat{\beta})^2 \frac{n}{\hat{\beta}^2} \\
W_2 &= (1 - \hat{\beta})^2 \frac{n}{\beta_0^2} = n(1 - \hat{\beta}^2)
\end{flalign*}
\\
Using $ \beta_0 = 1 $ \\
\\

\break

\subsection*{2.b}

Show that the score statistic, S, can be written \\
\\
\[ S = (1 - \hat{\beta} )^2 \ \frac{n}{\hat{\beta}^2} \]
\\
\textbf{Definition:} The Score test statistic is defined as \\
\begin{empheq}[box={\mymath[colback=blue!8,drop lifted shadow, sharp corners]}]{equation*}
  S = \frac{ \big\{ l'(\theta_0) \big\}^2 } {E \big\{-l''(\theta) \big\}_{|_{\theta_0}}}
\end{empheq}
\\
From question part 1.b
\\
\begin{flalign*}
  l'(\beta) &= \frac{n}{\beta} - \sum_{i=1}^{n} \ log \ x_i \\
\\
  l'(\beta_0 = 1) &= \frac{n}{\beta_0} - \sum_{i=1}^{n} \ log \ x_i = n - \sum_{i=1}^{n} \ log \ x_i
\end{flalign*}
\\
\[ E \big\{-l''(\beta) \big\}|_{\beta_0=1} = \frac{n}{\beta_0^2} = n \]
and
\\
\[ \hat{ \beta } = \frac{n}{\sum_{i=1}^{n} \ log \ x_i } \]
\\
\[ \sum_{i=1}^{n} \ log \ x_i = \frac{n}{\hat{ \beta }} \]
\\
\begin{flalign*}
  l'(\beta) &= n - \frac{n}{\hat{ \beta }} \\
  l'(\beta) &= n (1 - \frac{1}{\hat{ \beta }})
\end{flalign*}
\\
\[ (l'(\beta))^2 = (n - \frac{n}{\hat{ \beta }})^2 = n^2(1 - \frac{1}{\beta})^2 \]
\\
\[ S = n^2 \frac{ (1 - \frac{1}{\hat{ \beta }})^2 }{n} = n(1 - \frac{1}{\hat{ \beta }})^2 \]
\\
Which we can rewrite using \\
\\
\[ (1 - \frac{1}{\beta})^2 = (1 +\frac{1}{\beta^2} - \frac{2}{\hat{\beta}} ) = \frac{1}{\hat{\beta^2}}( \hat{\beta^2} + 1 - 2\hat{\beta} ) = \frac{1}{\hat{ \beta }^2} (1 - \hat{\beta})^2 \]
\\
Therefore
\\
\[ S = ( 1 - \hat{\beta} )^2 \frac{n}{\hat{ \beta }^2} \]
\\
As required.


\break

\subsection*{2.c}

The common procedure to evaluate the tests and obtain the appropriate p-value would be: \\
\\
1 - Evaluate the Wald $ W_1, W_2 $ or Score S test statistic value, r say, using the previously derived test statistic formulae and observed sample data \\
\\
2 - Calculate the p-value from r using the appropriate asymptotic null distribution, in this case this is $ \chi^2(d=1) $
\begin{itemize}
  \item The p-value is given by $ p = P(Y \geq r) $, where $ Y \sim \chi^2(1) $ and r is the observed value of $ 2 log(LR) $ which in our case is approximated by our test statistics r
  \item The p-value $ P(X^2(d=1) \geq r) $ would be $ 1 - P(\chi^2(d=1) < r) $ or 1-F(r) where F is the cdf of $\chi^2(d=1) $, the asymptotic null distribution
\end{itemize}
For example, if r was 3.84 we would find the p-value = $ \chi^2_{d=1}(x \geq 3.84) = 1 - \chi^2_{d=1}(x < 3.84) = 1 - 0.95 = 0.05 $ (this is the probability of getting a result at least this extreme, if the null hypothesis is true) \\
\\
3 - Decide whether to accept or reject the null hypothesis by comparing the p-value against a probability threshold $ \alpha $:
\begin{itemize}
  \item Lower p-values mean the test statistic value is more unlikely if the null hypothesis is true, so we have more evidence to reject the null hypothesis $ H_0 $, so we might reject the null hypothesis as false, more readily.
  \item Higher p-values mean the test statistic value is more likely if the null hypothesis is true, so we have less evidence to reject the null hypothesis  $H_0$, we might accept the null hypothesis as true, more readily.
\end{itemize}
\vspace{5mm}
The justification for the using the $ \chi^2(d = 1) $ is due to the fact that the asymptotic null distribution of the likelihood ratio (LR) test statistic approaches $ \chi^2(d) $, where d is the degrees of freedom difference between the null and alternative hypothesis. \\
\\
i.e. $ 2 log(LR) \approx \chi^2(d) $, if the null hypothesis $ H_0 $ is true \\
\\
(This approximation improves as the sample size n increases) \\
\\
In our example, we have $ H_0 $ with zero free parameters so $ d_0 = 0 $ and $ H_1 $ with 1 free parameter $ \beta_1 $ so $ d_1 = 1 $. Therefore d = 1 in our case, since there is 1 degree of freedom difference between the null $ H_0 $ and alternative $ H_1 $ hypothesis. $ d = d_1 - d_0 = 1 - 0 = 1 $ \\
\\
So the $ \chi^2(d=1) $ distribution is the correct distribution to use to compare with the test statistic of the observed sample data. \\
\\
If however, we started out with a fixed-level test size in mind, a similar approach using the appropriate quantile of the same $ \chi^2(d=1) $ distribution would be: \\
\\
Compare the observed value of the test statistic r with the $ \chi^2(1) $ distribution. Similarly, reject the null hypothesis $ H_0 $ if the observed value of the test statistic r is too large. For example, a fixed-level test of size $ \alpha = 0.05 $ will reject $ H_0 $ if the test statistic value r is greater than the $ 1 - 0.05 = 0.95^{th}$-quantile of the $ \chi^2(1) $ distribution, which has a value of 3.84.


\break

Further detail (also for my revision/understanding purposes!) \\
\\
1 - Calculate the test statistic T from the observed values.
\begin{itemize}
  \item If the null hypothesis is true, these test statistics T approximate 2 log(LR) from the observed data. They are equivalent asymptotically to the likelihood ratio test, as the sample size grows large its asymptotic null distribution is $ \chi^2(d) $
\end{itemize}
\vspace{5mm}
2 - Use the appropriate $ \chi^2(d) $ and find the critical value k to be compared with the test statistic.
\begin{itemize}
  \item $ k = \chi^2_{1-\alpha}(d) $, where $ \chi^2_{1-\alpha} $ denotes the $ (1 - \alpha) $-quantile of the $ \chi^2(d) $ distribution
  \item An example would be for a test size $ p = 0.05 $ we would find the $ 1 - p = 0.95 $ quantile of the $ \chi^2(d) $ distribution
  \item Where the p-value is the probability of observing data at least as extreme as those obtained, if the null hypothesis were true. Hence, p-value = $ P(\chi(d)^2 \geq 2 log(LR_0) | H_0 = true) $ where $ log(LR_0) $ is the observed value of the log-likelihood ratio test statistic.
\end{itemize}
\vspace{5mm}
3 - Compare the test statistic value r with the critical value k from the selected quantile of $ \chi^2(d = 1) $
\begin{itemize}
  \item If the test statistic is too large, reject the null hypothesis $ H_0 $
  \item If the test statistic is not too large, we accept the null hypothesis $ H_0 $
\end{itemize}
\vspace{5mm}
\textbf{Definition:} The p-value is defined as \\
\\
\begin{empheq}[box={\mymath[colback=blue!8,drop lifted shadow, sharp corners]}]{equation*}
  p=P(log(LR) \geq log(LR_0) | H_0 \ true)
\end{empheq}
\\\\
The p-value is given by $ p = P(Y \geq r) $ , where $ Y \approx \chi^2(d) $ and r is the observed value of 2 log(LR) (in our case the value of the test statistic approximation thereof).


\break 

\section*{Question 3}

\subsection*{3.a}

Given \\
\[ X_n = X + \frac{Z_n}{n^{1/2}} \] \\
and
\[ E(Z_n^2) = cn^{\alpha} \text{ ; where }  c > 0, \alpha \in \mathbb{R} \] \\
\\
Using Chebyshev's inequality, show that
\\
\[ P(|X_n - X| > \epsilon) \leq \frac{c}{\epsilon^2 n^{1-\alpha}} \] \\
\\
\textbf{Definition:} Chebyshev's inequality is defined as: \\
\\
\begin{empheq}[box={\mymath[colback=blue!8,drop lifted shadow, sharp corners]}]{equation*}
  P(|X| \geq a) \leq \frac{ E(X^2) }{a^2}
\end{empheq}
\\
So we have
\begin{flalign*}
P(|X_n - X| > \epsilon) &= P(|\frac{Z_n}{n^{1/2}}| > \epsilon) \\
\\
&= P(| Z_n | > \epsilon n^{1/2} ) \\
\\
& \leq P(| Z_n | \geq \epsilon n^{1/2} ) \\
\\
& \leq \frac{ E(Z_n^2)}{(\epsilon n^{1/2})^2} \text{ (using Chebyshev's inequality) } \\
\\
& = \frac{ cn^\alpha }{ (\epsilon n^{1/2} )^2} = \frac{ cn^\alpha }{\epsilon^2 n} \\
\\
& = \frac{ cn^\alpha n^{-1} }{\epsilon^2} = \frac{ cn^{\alpha-1}}{\epsilon^2} \\
\\
P(|X_n - X| > \epsilon) &= \frac{c}{ \epsilon^2 n^{1-\alpha} } \\
\end{flalign*}
\\
Therefore we have shown that
\\
\[ P(|X_n - X| > \epsilon) \leq \frac{c}{\epsilon^2 n^{1-\alpha}} \] \\
\\
As required. \\


\subsection*{3.b}

For what values of $ \alpha $ does the argument in part (a) prove that $ X_n $ converges in probability to $ X $ ? \\
\\
We have
\\
\[ P(| X_n - X | > \epsilon) \leq \frac{c}{ \epsilon^2 n^{1-\alpha} } \] \\
\\
We can conclude $ X_n $ tends to $ X $ by probability convergence (as $ n \rightarrow \infty $) when \\
\\
\[ P(| X_n - X | > \epsilon) \rightarrow 0 \]

\[ \frac{c}{ \epsilon^2 n^{1-\alpha} } \rightarrow 0 \] \\
\[ \frac{c}{ \epsilon^2 n^{1-\alpha} } = \frac{c}{\epsilon^2} n^{\alpha-1} \] \\
\\
As $ n \rightarrow \infty $ we require $ n^{\alpha-1} \rightarrow 0 $ \\
\\
$ n^k \rightarrow 0 $ as $ n \rightarrow \infty $ for $ k < 0 $  so for probability convergence we require \\
\\
$ \alpha - 1 < 0 $, therefore when $ \alpha < 1 $ we have $ X_n \rightarrow X $ by probability convergence. \\

\subsection*{3.c}

For the values of $ \alpha $ identified in part (b), what other mode of convergence of $ X_n $ to $ X $ is assured (without any further calculations)? \\
\\
We can use the rules of implication for convergence \\
\\
\textbf{Definition:}
\\
\begin{empheq}[box={\mymath[colback=blue!8,drop lifted shadow, sharp corners]}]{equation*}
  \{ X_n \xrightarrow{ms} X \} \implies \{ X_n \xrightarrow{p} X \} \implies \{ X_n \xrightarrow{D} X \}
\end{empheq}

We have shown probability convergence in part 3.b \\
\\
\[ \{ X_n \xrightarrow{p} X \} \implies \{ X_n \xrightarrow{D} X \} \]
\\
This implies also distribution convergence by the above definition of the implications of convergence. \\
\\

\end{document}  
% END DOCUMENT  % (end)
% ----------------------------------------------------------------------
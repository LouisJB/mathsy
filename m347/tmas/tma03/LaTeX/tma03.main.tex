\documentclass[11pt]{article}   	% use "amsart" instead of "article" for AMSLaTeX format
\usepackage{geometry}                		% See geometry.pdf to learn the layout options. There are lots.
\geometry{a4paper, margin=1in}                % ... or a4paper or a5paper or ... 
%\geometry{landscape}                		% Activate for for rotated page geometry

%\usepackage{enumitem}
%\parskip = \baselineskip
%\setlength{\parskip}{5pt}
%\setlength{\parindent}{0pt}
\usepackage[parfill]{parskip}    			% Activate to begin paragraphs with an empty line rather than an indent

%\usepackage{enumitem}

\usepackage{graphicx}				% Use pdf, png, jpg, or eps§ with pdflatex; use eps in DVI mode

\usepackage{amsfonts}								% TeX will automatically convert eps --> pdf in pdflatex		
\usepackage{amssymb}
\usepackage{amsmath}

\usepackage{hyperref}

\setlength\parindent{0pt}			% globally suppress indentation

\usepackage{empheq}
\usepackage[most]{tcolorbox}

\usepackage{changepage}   % for the adjustwidth environment

\newtcbox{\mymath}[1][]{%
    nobeforeafter, math upper, tcbox raise base,
    enhanced, colframe=blue!30!black,
    colback=blue!30, boxrule=1pt,
    #1}

\newenvironment{monoblock}%
  {\ttfamily \small}%
  {}
  
\newenvironment{tight_enumerate}{
\begin{enumerate}
  \setlength{\itemsep}{0pt}
  \setlength{\parskip}{0pt}
}{\end{enumerate}}


\title{M347 TMA03 R2698663}
\author{Louis Botterill}
\date{04-March-2018}					% Activate to display a given date or no date


\begin{document}
\maketitle

\pagebreak

\section*{Question 1}

\subsection*{1.a}

\[ f( x_i | \theta ) = \theta^2 x_i e^{- \theta x_i } \text{ on } x_i > 0 \] 
\\
if we take logs we can equate this to a standard natural exponential family NEF. \\
\
\[ log ( f( x_i | \theta) ) = log(\theta^2) + log(x_i) + - \theta x_i \]
\\
The standard form of the NEF is \\
\
\[ log ( f( x | \theta) ) = x a( \theta ) - b( \theta ) + c(x) \]
\
So we have \
\
\begin{flalign*}
b(\theta) &= -log(\theta^2) \\
c(x_i) &= log(x_i) \\
a(\theta) &= -\theta
\end{flalign*}
\\
Now we can make use of the standard result for the NEF, \
\begin{flalign*}
L(\theta) & \propto e^{n \bar{x} a(\theta) - n b(\theta) } \\
\
& \propto e^{ - n \bar{x} \theta - n (-log ( \theta^2 ) ) } \\
& \propto e^{ - n \bar{x} \theta + n log ( \theta^2 ) } \\
& \propto e^{ - n \bar{x} \theta } e^{n log ( \theta^2 ) } \\
& \propto e^{ - n \bar{x} \theta } e^{log ( \theta^{2n} ) }
\end{flalign*}
\
Therefore we obtain \
\
\[ L(\theta) \propto \theta^{2n} e^{-\theta n \bar{x} } \]
\
as required. \\


\subsection*{1.b}

The posterior density is given by: \\
\
\[ f( \theta | x ) = \frac { f( x | \theta ) f( \theta ) } { f(x) } \] \\
\
With a $ gamma(a, b) $ prior we have \\
\
\[ f( \theta) = gamma(a, b) \propto \theta^{ a - 1 } e^{ -b \theta } \] \\
\
ignoring constant terms (the normalisation constant $ f(x) $), we then obtain \\
\\
Posterior $ \propto $ Likelyhood x Prior \\
\\
\[ f( \theta | x ) \propto f( x | \theta ) f( \theta ) \]
\
\[ f(x | \theta) = L(\theta) \]
\
\[ \theta^{2n} e^{-\theta n \bar{x} } \times  \theta^{ a - 1 } e^{ -b \theta } \]
\
Grouping terms \\
\
\[ \theta^{2n} \theta^{ a - 1 } e^{-\theta n \bar{x} } \times e^{ -b \theta } \]
\
\[ \theta^{2n + a - 1 } e^{-\theta ( n \bar{x} + b ) } \]
\
Therefore, the posterior can be written as \\
\
\[ f( \theta | x ) \propto \theta^{a+2n-1} e^{-(b+n \bar{x}) \theta } \]
\
as required. \\


\subsection*{1.c}

The posterior corresponds to the $ Gamma(a + 2n, b + n \bar{ x } ) $ distribution. \\


\subsection*{1.d}

Given: \\
\\
$ E[ \theta ] = \frac{a}{b} $ \\
\\
MLE $ \hat{\theta} $ of $ \theta $ under $ gamma(2, \theta) = \hat{\theta} = \frac{2}{ \bar{x} } $ \\
\\
Posterior mean is: \\
\
\[ E[ \theta | x ] = \frac { a + 2n }{ b + n \bar{x} } \]
\
Expanding the posterior mean we have \\
\
\[ E[ \theta | x ] = \frac { a }{ b + n \bar{x} } + \frac { 2n }{ b + n \bar{x} } \]
\
To write this in the form \\
\
\[ E[ \theta | x ] = t \hat{\theta} + (1 - t) E[ \theta ] \text{ over } 0 < t < 1 \]
\
Start with \\
\
\[ E[ \theta | x ] = t \frac{2}{\bar{x}} + (1 - t) \frac{a}{b} \]
\
We can equate the terms, the 2nd term can be written \\
\
\[ t \frac{2}{\bar{x}} = \frac{ n \bar{x} }{ b + n \bar{x} } \times \frac{2}{\bar{x}} = \frac { 2 n }{ b + n \bar{x} } \]
\\
So t is therefore \\
\
\[ t = \frac{ n \bar{x} }{ b + n \bar{x} } \]
\\
Similarly then for (1 - t) we have \\
\\
\[ 1 - t = \frac{ b + n \bar{x} - n \bar{x} }{ b + n \bar{x} } = \frac{b}{b + n \bar{x} } \]
\\
and indeed this equates to the first term
\\
\[ (1 - t) \frac{a}{b} = \frac{ b }{ b + n \bar{x} } \frac{a}{b} = \frac{a}{ b + n \bar{x} } \]
\\
So we've shown we can write the posterior in the form \\
\
\[ E[ \theta | x ] = t \frac{2}{\bar{x}} + (1 - t) \frac{a}{b} \]
\\
Therefore $E(\theta|x) $ can be written as \\
\
\[ E[ \theta | x ] = t \hat{\theta} + (1 - t) E[ \theta ] \text{ over } 0 < t < 1 \]
\\
as required.

\break 


\section*{Question 2}

\subsection*{2.a}

\[ E[L(d, \theta) | x] = 1 + F(d - \alpha | x) - F(d + \alpha | x) \] 
\
\[ F(\theta | x) = 1 - F(2 \theta_0 - \theta | x) \]
\
\[ E[L(d, \theta) | x] = 1 + F (d - \alpha | x) - 1 - F(2 \theta_0 - (d + \alpha) | x) \] 
\
\[ E[L(d, \theta) | x] = F (d - \alpha | x) - F(2 \theta_0 - (d + \alpha) | x) \]
\


\subsection*{2.b}

\[ E[L(d, \theta) | x] = F (d - \alpha | x) - F(2 \theta_0 - (d + \alpha) | x) \]
\
recall that \\
\
\[ F(d|x) = \int_{\infty}^d f(\theta | x) d \theta \]
\
so that \
\
\[ \frac{d}{dd} F(d|x) = \frac{d}{dd} \int_{\infty}^d f(\theta | x) d \theta = f(\theta|x) \]
\
\[ \frac{d}{dd} E[ L(d, \theta) | x ] = f(d - \alpha | x) - f( 2 \theta_0 - d - \alpha | x) \]
\
when $ d = \theta_0 $ \\
\
\[ \frac{d}{dd} E[ L(d, \theta) | x ] = f(d - \alpha | x) - f( 2 \theta_0 - \theta_0 - \alpha | x) \]
\
\[ \frac{d}{dd} E[ L(d, \theta) | x ] = f(d - \alpha | x) - f( \theta_0 - \alpha | x) \]
\
\[ \frac{d}{dd} E[ L(d, \theta) | x ] = f(d - \alpha | x) - f( d - \alpha | x) \]
\
\[ \frac{d}{dd} E[ L(d, \theta) | x ]  = 0 \]


\subsection*{2.c}

\[ \frac{d^2}{dd} E[ L(d, \theta) | x ] = \frac{d}{dd} f(d - \alpha | x) - f( 2 \theta_0 - d - \alpha | x) \]
\\
Using the chain rule for differentiation of nested functions, which is \\
\
\[ k(x) = g(f(x)) \text{ then } k'(x) = g' ( f(x) ) \times f'(x)  \]
\
\[ \frac{d}{dd} f(d - \alpha | x)  = 1 \times f'(d - \alpha | x) \]
\
\[ \frac{d}{dd} -f(2 \theta_0 - d - \alpha | x) = -1 \times - f'(2 \theta_0 - d - \alpha | x) = f'(2 \theta_0 - d - \alpha | x) \]
\\
So we have \\
\
\[ \frac{d^2}{dd} E[ L(d, \theta) | x ] = f'(d - \alpha | x) + f'( 2 \theta_0 - d - \alpha | x) \]
\\
at $ d = \theta_0 $ when then get \\
\
\[ \frac{d^2}{dd} E[ L(d, \theta) | x ] = f'(\theta_0 - \alpha | x) + f'( 2 \theta_0 - \theta_0 - \alpha | x) \]
\
\[ \frac{d^2}{dd} E[ L(d, \theta) | x ] = 2f'(\theta_0 - \alpha | x) \]
\\
At $ \theta = \theta_0 - \alpha $ we know $ \theta < \theta_0 $ always (since $ \alpha > 0 $) so, using the differentiation properties of the posterior density $ f(\theta|x) $ as given, \\
\
\[ f'(\theta | x) > 0 \text{ for } \theta < \theta_0 \]
\
Therefore \\
\
\[ 2 f'(\theta_0 - \alpha | x) > 0 \]
\
and so \\
\
\[ \frac{d^2}{dd} E[ L(d, \theta) | x ] > 0 \]
\\
So we have shown that $ \frac{d^2}{dd} E[ L(d, \theta) | x ] $ is positive at $ d = \theta_0 $, as required (for any $ \alpha > 0 $).\\


\subsection*{2.d}

In this question we have shown that the $ 1^{st} $ derivative of the expected loss function is 0 (therefore a stationary point at $ d = \theta_0 $) and that the $ 2^{nd} $ derivative of the expected loss about the mode of the posterior density $ d = \theta_0 $ is positive, showing that $ d = \theta_0 $ is a minimiser of the expected loss. \\
Furthermore we are told that $ \theta_0 $ is the mode of the posterior. So the minimiser of the expected loss $ E[L(d, \theta) | x] $ is at the mode of the posterior at $ \theta_0. $ \\
Also since the posterior is unimodal, this is the only minimiser, it is therefore the global minimiser of the expected loss function. We have made this proof under the basis $ \alpha > 0 $ and so $ \alpha $ is also non-zero. We don't know the behaviour of derivative of the posterior density at $ \theta_0 $ so we can not extend the proof to $ alpha = 0 $. \\
\\
So the optimal estimator of $ \theta $ for the symmetric posterior under 0-1 loss function is the posterior mode, for positive $ \alpha $. \\
\\

\break 


\section*{Question 3}

\subsection*{3.a}

We have \
\
\[ \mu^{*}_{2} \sim N(\mu^\prime_1, 2) \]
\[ \mu^\prime_1 = 0 \]
\\
Therefore the Metropolis-Hastings algorithm will use the following distribution to generate candidate value $ \mu^* $ for $ \mu_2 $ \\
\
\[ \mu^{*}_{2} \sim N(0, 2) \]


\subsection*{3.b}

The general acceptance probability for Metropolis-Hastings sampling is \
\\
\[ \alpha(\mu^* | \mu_t) = min \Big( \frac { q(\mu_t | \mu^*) f(\mu^*) } { q(\mu^* | \mu_t) f(\mu_t)}, 1 \Big) \]
\\
This can be simplified under a symmetric proposal density where \\
\
\[ q(\mu_{t + 1} | \mu_t) = q(\mu_t | \mu_{t + 1}) \] \\
\
The proposal density is $ N \sim (\mu = \mu_t, \sigma^2 = 2) $ \\
\\
Where the Normal distribution is defined as \\
\
\[ f(x) = \frac{ 1 }{ \sigma \sqrt{ 2 \pi } } e^{ -\frac{1}{2} ( \frac{x - \mu}{\sigma} )^2 } \text{ on } \mathbb{R} \]
\\
So ignoring linear constants of proportionality we have \\
\
\[ q(\mu_{t + 1} | \mu_t) \propto e^{ - \frac{1}{2x2} (x_{t+1} - x_t)^2 } \]
\
\[ q(\mu_t | \mu_{t + 1}) \propto e^{ - \frac{1}{2x2} (x_t - x_{t + 1} )^2 } \] \\
\
since $ (x_{t+1} - x_t)^2 = (x_t - x_{t + 1} )^2 $ \\
\\
these two expressions are the same and therefore cancel out, leaving \\
\
\[ \alpha(\mu^* | \mu_t) = min \Big( \frac { f(\mu^*) } { f(\mu_t)}, 1 \Big) \] \\
\
So when X was observed to be 1 we have for the simulated value of $ \mu^\prime_{t=1} $ the candidate acceptable probability of \\
\
\[ \alpha(\mu^* | \mu^\prime_1) = min \Big( \frac { f(\mu^* | x = 1) } { f(\mu^\prime_1 | x = 1)}, 1 \Big) \] \\
\
as required.


\subsection*{3.c}

Given \\
\
\[ \mu^\prime_1 = 0 \]
\
\[ \mu^* = 1.2 \]
\
\[ f(\mu | x = 1) \propto (1 + e^{-(\mu+2)})^{-1} e^{-\mu^2 / 2} \] \
\
So we have
\\
\[ f(\mu^* | x = 1) \propto (1 + e^{-(1.2+2)})^{-1} e^{-1.2^2 / 2} = (1 + e^{-3.2})^{-1} e^{-0.72} \]
\
\[ f(\mu^* | x = 1) \propto 0.9608 \times 0.4868 = 0.4677 \]
\
\[ f(\mu^{\prime}_1 | x = 1) \propto (1 + e^{-(0+2)})^{-1} e^{0^2 / 2} = (1 + e^{-2})^{-1} = 0.8808 \]
\
Using \\
\
\[ \alpha(\mu^* | \mu_1^\prime) = min \Big( \frac { f(\mu^* | x = 1 ) } { f(\mu^\prime_1 | x = 1 )}, 1 \Big) \]
\
We get \\
\
\[ \alpha(\mu^* | \mu^\prime_1) = min \Big( \frac { 0.4677 } { 0.8808 }, 1 \Big) \]
\
\[ \alpha(\mu^* | \mu^\prime_1) = 0.531 \text{ (to 3 d.p.) } \]


\subsection*{3.d}

Since $ \alpha(\mu^* | \mu^\prime_t) = 0.531 < u \sim U(0, 1) = 0.81 $ this candidate value is accepted. \\
\\
Therefore the value of $ \mu^{\prime}_2 $ is updated to 1.2 \\
\\
Therefore the candidate $ \mu^{**} $ for $ \mu_3 $ will be drawn from $ \sim N(1.2, 2) $ which is the same normal distribution with mean updated to be centred around the latest accepted value of $ \mu $. \\
\\


\end{document}  
% END DOCUMENT  % (end)
% ----------------------------------------------------------------------
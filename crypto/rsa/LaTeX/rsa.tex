\documentclass[11pt]{article}   	% use "amsart" instead of "article" for AMSLaTeX format
\usepackage{geometry}                		% See geometry.pdf to learn the layout options. There are lots.
\geometry{a4paper, margin=1in}                % ... or a4paper or a5paper or ... 

\usepackage{graphicx}				% Use pdf, png, jpg, or eps§ with pdflatex; use eps in DVI mode
	
\usepackage{amssymb}

\usepackage{hyperref}

\usepackage{parskip}

\usepackage{enumitem}
\setitemize{noitemsep,topsep=0pt,parsep=0pt,partopsep=0pt}

\newenvironment{monoblock}%
  {\ttfamily \small}%
  {}
  
\newenvironment{tight_enumerate}{
\begin{enumerate}
  \setlength{\itemsep}{0pt}
  \setlength{\parskip}{0pt}
}{\end{enumerate}}
    
\newenvironment{boxed}
    {\begin{center}
    \begin{tabular}{|p{0.9\textwidth}|}
    \hline\\
    }
    { 
    \\\\\hline
    \end{tabular} 
    \end{center}
    }
 
\title{RSA for beginners - version 0.1}
\author{Louis Botterill}
\date{2025-04-04}					% Activate to display a given date or no date

\begin{document}
\maketitle

\pagebreak

\section*{1 RSA public key cryptography}

\subsection*{1.1 Introduction}

Nothing in this document is secret and it is all readily available in the public domain. The aim of this document is to consolidate it into one place in an easily digestible form, for beginners or as a refresher for those who are already familiar with it. \\
\\
RSA is a public (asymmetric) key cryptographic system. Let's see what that means and how it works.
Firstly, some background. There are two main types of cryptographic systems, these are symmetric and asymmetric (also known as public key) cryptography. \\
\\
\textbf{Symmetric} - uses one private key. Encryption and decryption are inverses, using the same privately shared key for both.
\begin{itemize}
 \item Advantages - speed and efficiency. Smaller keys for same cryptographic strength.
 \item Disadvantages - difficulty in distributing the private keys, without risk of compromising them.
\end{itemize}

\textbf{Asymmetric} - there are two keys, a public and a private key, mathematically related as a pair. Anyone may obtain the public key, but the private must be kept secret.
\begin{itemize}
 \item Advantages - key distribution is vastly simplified. Keys do not need to be distributed to both parties.
 \item Disadvantage - generally slower, computationally more expensive. Larger keys for same cryptographic strength.
\end{itemize}

For the purposes of describing the process of sending secure messages from a to b, we'll introduce the standard parlance, Alice and Bob. Alice is the sender and Bob is the recipient. Since we're discussing RSA here, which is a public asymmetric system, Bob will have a key pair, private and public. Only Bob will know his private key, but his corresponding public key is shared and Alice will know it and use it to send secure messages to Bob (that only Bob can decrypt). Because the private and public keys are related they form a secure encrypt-decrypt process together and are generated at the same time from common parameters. How exactly this is done will be worked through in following sections. \\
\\
This document will help take you through the necessary definitions, background mathematics and how it is used in RSA asymmetric public key cryptography. Further reading may be necessary but the aim is to have this relatively self contained in a single document intended for beginners, hopefully without being either too high or low-level.


\section*{1.2 RSA}

RSA is a popular cryptographic system, it stands for the authors Rivest-Shamir-Adleman who publicly described the algorithm in 1977.
It is a is a widely used public key (asymmetric) encryption algorithm. \\
\\
RSA uses a pair of keys – a public key for encryption and a private key for decryption and is based on modular arithmetic.
The scheme was originally invented by Clifford Cock while working at GCHQ in Great Britain, back in 1973. This project remained secret until 1997 and hence RSA took hold in the public domain in the meantime. \\
\\
The key premise of RSA public key cryptography is based on the following key mathematical result \\
\\
\textbf{The central formula} \\
\\
$ m^{e.d} \equiv m \ (mod \ n)  \ \ \underline{\hspace{2cm}}(1) $\\
\\
m - by convention we denote m as the message \\
e.d - the product of two integers, with a relation to Euler totient function which will be explained later \\
\\
To set this up, we need to choose \\
\\
e - a public exponent, typically a fixed known value - most commonly this is $ 2^{16} + 1 = 65537 $ \\
d - private value found such that $ e.d $ satisfy particular criteria to be explained below \\
n - a modulus made from the product of two large primes, p and q, called n \\
\\
Both n and e are components of the public key, which are published and shared. \\
A sender wishing to send a secret message m to a recipient encrypts it for that recipient using their (e, n) public key. \\
The recipient has the corresponding private key (d, n) to decrypt and recover m. \\
\\
This may or may not seem simple enough, to help understand how this works and is used this document will run through the relevant parts and supporting mathematics. \\
\\
Given all this, then we use it like this \\
\\
\textbf{To encrypt} \\
\\
To Encrypt a plain text message m for Bob, Alice requires Bob's public key (e, n) \\
\\
$ c = m^e \ mod \ n $ \\
\\
where c is the encrypted ciphertext and n is the public modulus, e the public exponent.
\\
The ciphertext c can now be transmitted safely in private from Alice to the recipient Bob over a public network. Only the intended recipient with the corresponding private key d can decrypt c back to m. Anyone else obtaining the ciphertext should not be able to decrypt it (easily, in theory). \\
\\
\textbf{To decrypt} \\
\\
To decrypt the ciphertext c of message m for Bob from Alice, this requires Bob's (matching) private key (d, n) \\
\\
$ m = c^d = (m^e)^d = m^{ed} \ (mod \ n) = m$ (the original message) by (1) \\

\break 


\textbf{Summary} \\
\\
we choose n = pq where p and q are very large primes \\
and e is selected as a fixed public 'exponent' \\
\\
\textbf{Public Key - (e, n)} \\
\\
    * Used for encrypting data (and verifying digital signatures). \\
    * Shared publicly, used to encrypt data only intended for the recipient. \\
    * comprised of the modulus (n) and the public exponent (e) as (e, n). \\
\\
n is the product of chosen large primes p and q \\
and e is a preselected exponent such as 65537 (a common value) \\
\\
\textbf{Private key - (d, n)} \\
\\
    * Used for decrypting data intended for the recipient. \\
    * Kept secret, not shared publicly, used to decrypt data only intended for the recipient. \\
    * comprised of the modulus (n) and the private value (d) as (d, n). \\
\\
\textbf{Details of e and d} \\
\\
The exact details do mater for this to work, e and d are carefully selected such that \\
\\
$ ed = k\phi(n) + 1 $ - where k is insignificant here \\
\\
i.e. \\
\\
$ ed \equiv 1 \  mod \ \phi(n) $ \\
\\
See section 2.7 for further explanation of how this actually works in more precise detail. \\
\\

\break 


\section*{2 Mathematical building blocks}

The following subsections break down some of the relevant key mathematical concepts and building blocks.

\subsection*{2.1 Prime numbers}

Most likely you know all about prime numbers, but as this is a beginners guide, let's briefly recap. \\
\\
Prime numbers are the numbers that have only 2 factors, 1 and themselves. In other words they are \textbf{only} divisible by 1 and themselves (note - as are all numbers, but for primes these are the only factors). Therefore they contain no other factors, thus they are further indivisible. To really qualify this we should note we're speaking of positive integers here, not real numbers and so on. \\
\\
The fundamental theorem of arithmetic (also known as the unique factorization theorem), states that every integer greater than 1 can be uniquely represented as a product of prime numbers. For some $ n > 1 $, where can express n as follows \\
\\
$ n = p_0^{e_0} \ p_1^{e_1} \ ... \ p_k^{e_k} $ \\
\\
where the prime factorization is unique. That is to say every integer greater than 1 has a unique prime factorization. Thus all integers have a unique representation as a decomposition of primes.


\subsection*{2.2 Modular arithmetic and congruence}

This is a system of arithmetic for integers, where numbers "wrap around" at a certain value, called the modulus. \\
\\
Example, for a modulus n say 3, the numbers modulo n (called mod n) are [0, 1, 2] (and no other numbers mod n are allowed) \\
\\
This can be thought of as the remainder $ < n $ after dividing out the original number by n \\
\\
When two numbers have the same remainder modulo a particular modulus, we say they are congruent. \\
\\
Example, 5 and 8 are both congruent to 2 mod 3 \\
\\
This would be written in mathematical notation as \\
\\
$ 5 \equiv 8 \equiv 2 \ mod \ 3 $ \\
\\
e.g. \\
\\
$ 5 - 3 = 2 $ and $ 8 - 2x3 = 8 - 6 = 2 $, thus both 5 and 8 are congruent to 2 mod 3 \\


\subsection*{2.3 co-primes and gcd}

The Greatest Common Divisor (GCD), of two or more non-zero integers, is the largest positive integer that divides both of the integers. GCD is also known as also known as Greatest Common Factor (GCF) or Highest Common Factor (HCF) \\
\\
We write g = gcd(a, b) where gcd is the greater common divisor number \\
\\
Two integers are coprime (also called relatively prime or mutually prime) if their greatest common divisor (GCD) is 1. \\
\\
In other words, they share no common factors other than 1.


\subsection*{2.4 Fermat’s little theorem (FLT)}

\boldmath $ a^p \equiv p \ mod \ n $ \unboldmath \\
\\
where p is prime and a is not divisible by p \\
therefore a and p are coprime i.e. gcd(a, p) = 1 \\
\\
a useful alternative formulation of this is \\
\\
$ a^{p-1} \equiv 1 \ mod \ n $ \\
\\
In either form, this is an important result, in general and for public key cryptography, such as in RSA.


\subsection*{2.5 Euler's Totient function}

Euler totient function for n counts the number of coprimes from 1 to n \\
\\
given some n has a prime factorization \\
\\
$ n = p_0^{e_0} \ p_1^{e_1} \ ... \ p_k^{e_k} $ \\
\\
The totient can be found using \\
\\
$ \phi(n) = (p_0 - 1)p^{e_0 - 1} \ (p_1 - 1)p^{e_1 - 1} \ (p_k - 1)p^{e_k - 1} $ \\
\\
when n is prime, p say \\
\\
$ \phi(p) = (p-1)p^0 = p - 1 $ \\
\\
Relevant to RSA, when n is a product of 2 primes, p and q, we have \\
\\
$ \phi(n) = \phi(p)\phi(q) = (p-1)(q-1) $ (which we note, is trivial to compute given we know the only prime factors of n, p and q) \\
\\
This fact is taken advantage of to make a large number n with an easily known $ \phi(n) $
\\
incidentally, the cototient of n is defined as $ n - \phi(n) $ i.e. the number of factors of n \\


\subsection*{2.6 Euler’s generalization of Fermat’s Little Theorem}

$ a^{\phi(n)} \equiv 1 \ mod \ n $ \\
\\
note here we're extending Fermat's little theorem to $ \phi(n) $ of the modulus n \\
\\
Furthermore whilst it would be hard to calculate $ \phi(n) $ for large arbitrary n, because we created it from two large (secret) primes, we can calculate it trivially as explained earlier.


\subsection*{2.7 RSA mathematics using these building blocks}

Recall we chose ed with special conditions \\
\\
$ ed = \phi(n) + 1 $ \\
\\
or more generally \\
\\
$ ed = k\phi(n) + 1 $ - where k is insignificant here \\
\\
i.e. \\
\\
$ ed \equiv 1 \  mod \ \phi(n) $ \\
\\
This works because \\
\\
$ m^{k \phi(n)} = m^{\phi(n)} . m^{\phi(n)} · . . . · m^{\phi(n)} = 1 \ (mod \ n) $ \\
\\
$ m^{ed} = m^{k\phi(n) + 1} = m . m^{k \phi(n)} = 1.m \ (mod \ n) = m \ (mod \ n) $ \\
\\
Another important aspect of this is that $ \phi(n) $ is hard to calculate from n (unless you know the prime factors of n). This is why n is calculated as the product of 2 large prime numbers. \\
\\
If we did know or were able to easily calculate $ phi(n) $ then we could easily compute d the private key \\
\\
$ d = (\phi(n) + 1) / e $ where everything on the right hand side is publicly available, so d would be easily compromised from publically available data. \\
\\
To ensure this is not the case, n is calculated from the product of 2 large primes, p and q. This means if you know the primes, it is easy to calculate $ \phi(n) = (p-1)(q-1) $, however if you don't (and these should not be shared) then it is very hard to find them (factorizing for large n and/or find $ \phi(n) $ by any other means, is hard. \\
Thus this is an important part of making RSA public key cryptography secure against compromises. 

\break


\section*{3 Summary}

In this short document we have seen how the popular public key (asymmetric) cryptographic system known as RSA can be used and how it works, using the simple mathematical concepts and building blocks. Key building blocks of modular arithmetic, prime numbers, Fermat's Little Theorem (FLT) and Euler's Generalization of FLT have been introduced along the way to provide an understanding of the mathematical mechanisms by which this system works.

\break


\section*{Further reading}

A Method for Obtaining Digital Signatures and Public-Key Cryptosystems - R.L. Rivest, A. Shamir, and L. Adleman \\
\\
\url{https://people.csail.mit.edu/rivest/Rsapaper.pdf} \\
\\
Wikipedia on RSA \\
\\
\url{https://en.wikipedia.org/wiki/RSA_cryptosystem} \\
\\
GCHQ \\
\\
\url{https://en.wikipedia.org/wiki/GCHQ} \\
\\
ITU \\
\\
\href{https://www.itu.int/en/ITU-D/Cybersecurity/Documents/01-Introduction%20to%20Cryptography.pdf}{ITU Introduction to cybersecurity} \\
\\
Stanford \\
\\
\url{https://crypto.stanford.edu/~dabo/cryptobook/draft_0_3.pdf} \\
\\
NIST Special Publication 800-175B \\
\\
\url{https://nvlpubs.nist.gov/nistpubs/SpecialPublications/NIST.SP.800-175Br1.pdf}

\break


\section*{Glossary}

Coprime - two numbers sharing no common factors other than 1. Two numbers are considered coprime if their greatest common divisor (GCD) or highest common factor (HCF) is 1 \\
\\
Chinese hypothesis - The hypothesis that an integer n is prime iff it satisfies the condition that $ 2^n - 2 $ is divisible by n \\
\\
FTL - Fermat's Little theorem. This is a generalization of the Chinese hypothesis and a special case of Euler's totient theorem. \\
\\
GCD - greatest common divisor. The highest divisor common to two numbers \\
\\
GCHQ - Government Communications Headquarters (GCHQ) \\
\\
HCF - see GCD \\
\\
Iif - short-form of if and only if \\
\\
Integer - whole numbers, 1, 2, 3 .. n \\
\\
Integer factorization - decomposition of a positive integer into a product of integers \\
\\
Modular arithmetic - wrap-around integers from 0 to $ n - 1 $ \\
\\
Mutually prime - see coprime \\
\\
Prime factorization - reduction to the unique factors of any number \\
\\
Prime number - numbers with no divisors (other than 1 and themselves) \\
\\
Relatively prime - see coprime \\
\\
Totient - for n, it is the number of coprimes from 1 to n \\

\break


\section*{Recommended further resources}

https://mathworld.wolfram.com/RSAEncryption.html \\
\\
https://mathworld.wolfram.com/GreatestCommonDivisor.html \\
\\
https://mathworld.wolfram.com/TrapdoorOne-WayFunction.html


\end{document}  
% END DOCUMENT  % (end)
% ----------------------------------------------------------------------